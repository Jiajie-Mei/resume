% !TEX TS-program = xelatex
% !TEX encoding = UTF-8 Unicode
% !Mode:: "TeX:UTF-8"

\documentclass{resume}
\usepackage{zh_CN-Adobefonts_external} % Simplified Chinese Support using external fonts (./fonts/zh_CN-Adobe/)
% \usepackage{NotoSansSC_external}
% \usepackage{NotoSerifCJKsc_external}
% \usepackage{zh_CN-Adobefonts_internal} % Simplified Chinese Support using system fonts
\usepackage{linespacing_fix} % disable extra space before next section
\usepackage{cite}

\newlist{InlineList}{enumerate*}{1}
\setlist[InlineList]{label=(\roman*)}


\begin{document}
\pagenumbering{gobble} % suppress displaying page number

\name{梅家洁}

\basicInfo{
  \email{jiajie.mei@buaa.edu.cn} \textperiodcentered\ 
  \phone{(+86) 188 1004 2048} %\textperiodcentered\ 
  %\linkedin[billryan8]{https://www.linkedin.com/in/billryan8}
  }
 
\section{教育背景}
\datedsubsection{\textbf{北京航空航天大学}}{2017.09 -- 2020.01}
\textit{硕士}\ 生成模型与知识图谱(mentor:张日崇,Yongyi Mao)
\datedsubsection{\textbf{北京航空航天大学}}{2013.09 -- 2017.07}
\textit{学士}\ 计算机科学与技术


\section{科研经历}
\datedsubsection{\textbf{计算机新技术研究所}, 北航}{2016.08 -- 2018.12}
\begin{itemize}
	\item 知识图谱
\begin{itemize}

\item 链接预测任务所使用的top-$k$评估准则难以让模型在所有的链接预测任务上都取得比较好的查准率和查全率。我们提出使用max-$k$准则,并提出了采样协议和贪心协议。实验证明,max-$k$能比top-$k$有更好的F1表现和查准率表现。该工作发表在SIGIR2018(On Link Prediction in Knowledge Bases: Max-K Criterion and Prediction Protocols)上(第一作者)。
\end{itemize}
\item 生成模型
\begin{itemize}
	\item 深入了解VAE,基于它做了以下研究:
	\begin{InlineList}

		\item 初步研究KL散度为零问题。我构造了一个最简单的VAE,通过推导目标函数的最优解,发现确实KL散度为零能导致极值。
  		\item 将VAE用于知识库表示学习,并投稿AAAI2018,未录用。
	\end{InlineList}
\item 了解GAN。我构造了一个玩具数据集,使用多个生成器与一个判别器对抗,学习数据的边界。
\end{itemize}


\end{itemize}

\section{实习经历}
\datedsubsection{\textbf{NLP中心}, 美团点评}{2018.12 -- 2019.03}
\begin{itemize}
\item 推荐理由生成(mentor:王金刚,张富峥)
\begin{itemize}
	\item 基于商户的用户评论以及用户查询,生成商户的推荐理由,辅助用户决策。该工作将用户查询通过BiLSTM编码,并且将该编码引入到Pointer-generator的编码器和解码器。实验结果表明该模型相比baseline更能生成查询相关的推荐理由生成,并且在ROUGE指标下取得了最好的结果。该工作已投稿IJCAI2019 (Query-Aware Tips Generation for Vertical Search)。
\end{itemize}


\end{itemize}


\section{技能}
% increase linespacing [parsep=0.5ex]
\begin{itemize}[parsep=0.5ex]
  \item 编程语言: Python > C/C++ > Java
  \item 平台: Linux
  \item 机器学习框架: Tensorflow
  \item 语言: 英语 - 熟练(四级600, 六级581, TOEFL 95)

\end{itemize}

\section{获奖情况}
\datedline{研究生国家奖学金}{2018.09}
\datedline{北京市优秀毕业生}{2017.07}
\datedline{第七届蓝桥杯大赛北京赛区C/C++程序设计大赛A组二等奖}{2016.03}
\datedline{2014-2015学年校级三好学生}{2015.10}
\datedline{2013-2014学年国家励志奖学金}{2014.12}

%% Reference
%\newpage
%\bibliographystyle{IEEETran}
%\bibliography{mycite}
\end{document}
