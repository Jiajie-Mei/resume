% !TEX TS-program = xelatex
% !TEX encoding = UTF-8 Unicode
% !Mode:: "TeX:UTF-8"

\documentclass{resume}
\usepackage{zh_CN-Adobefonts_external} % Simplified Chinese Support using external fonts (./fonts/zh_CN-Adobe/)
% \usepackage{NotoSansSC_external}
% \usepackage{NotoSerifCJKsc_external}
% \usepackage{zh_CN-Adobefonts_internal} % Simplified Chinese Support using system fonts
\usepackage{linespacing_fix} % disable extra space before next section
\usepackage{cite}

\newlist{InlineList}{enumerate*}{1}
\setlist[InlineList]{label=(\roman*)}


\begin{document}
\pagenumbering{gobble} % suppress displaying page number

\name{梅家洁}

\basicInfo{
  \email{jiajie.mei@buaa.edu.cn} \textperiodcentered\ 
  \phone{(+86) 188 1004 2048} %\textperiodcentered\ 
  %\linkedin[billryan8]{https://www.linkedin.com/in/billryan8}
  }
 
\section{教育背景}
\datedsubsection{\textbf{北京航空航天大学}}{2017.09 -- 2020.01}
\textit{硕士}\ 生成模型与知识图谱
\datedsubsection{\textbf{北京航空航天大学}}{2013.09 -- 2017.07}
\textit{学士}\ 计算机科学与技术,保研综合排名29/238


\section{科研经历}
\datedsubsection{\textbf{计算机新技术研究所 (ACT)}, 北航,导师:张日崇\& Yongyi Mao}{2016.08 -- 2018.12}
\begin{itemize}
	\item 知识图谱:{\bf Jiajie Mei}, Richong Zhang, Yongyi Mao, Ting Deng. On Link Prediction in Knowledge Bases: Max-K Criterion and Prediction Protocols (SIGIR2018)
	\begin{itemize}
		\item 观察到top-$k$准则不能在所有的实体预测任务上都取得均衡的查准率 (P) 和查全率 (R)。
		\item 提出使用max-$k$准则,并指出对于理想模型,max-$k$准则的P, R, F1至少和top-$k$的一样好。
		\item 提出了一个符合max-$k$准则的预测协议,即采样协议。
		\item 通过对采样协议进行渐近分析,导出了另一个max-$k$协议——贪心协议。
	\end{itemize}
\item 生成模型(探索性):
\begin{itemize}
	\item 深入理解Variational AutoEncoder (VAE), 初步研究过VAE在训练时出现KL散度为零问题。我尝试推导线性VAE优化目标的最优解,推导得到KL为零所对应的解确实是一个驻点。
	\item 了解Generative Adversarial Network (GAN),曾经想基于它构造一个句子打分器。在初步研究中,构造了一个二维玩具数据集,使用一个判别器与多个生成器对抗,学习数据边界。但是GAN的优化目标内在地决定了最终生成器会打败判别器,这个思路行不通。
\end{itemize}
% \begin{InlineList}
% \item 我,基于它做了一些探索性研究。例如,我初步研究过VAE用于句子生成时常会出现KL散度为零问题。相比现有的一些经验性方法,我尝试推导VAE优化目标的最优解。为此,我构造了一个最简单的线性VAE,推导得到KL为零所对应的解确实是一个驻点。
% \item 我了解GAN,基于它也做过一些探索性研究。我曾经想基于GAN构造一个句子打分器。在初步研究中,我构造了一个二维的玩具数据集,使用多个判别器与一个生成器对抗。但是最终生成器总是能近似生成玩具数据,判别器会失效,达不到打分的目的。后来我意识到,这是GAN的优化目标必定会导致的情况。
% \end{InlineList}


\end{itemize}

\section{实习经历}
\datedsubsection{\textbf{NLP中心}, 美团点评,导师:王金刚\&张富峥}{2018.12 -- 2019.03}
\begin{itemize}
\item 推荐理由生成:Query-Aware Tips Generation for Vertical Search(已投稿IJCAI2019)
\begin{itemize}
\item 使用Hive, Hadoop, Spark等工具处理点评的搜索日志,进行数据清洗,导出数据集。
\item 尝试了Pointer-generator模型(用户评论作为原文,推荐理由作为摘要),并在encoder端引入selective gate进一步提升ROUGE性能。
\item 为了让推荐理由与查询相关,将用户查询通过BiLSTM编码,将其引入到encoder和decoder中。该模型达到最好的自动指标表现 (ROUGE-1/2/L的F1为69.08/63.43/68.29)和人工评测表现(89\%的准确度,3.97/5的可读性,3.17/5的查询相关性)。
\end{itemize}
\end{itemize}


\section{技能}
% increase linespacing [parsep=0.5ex]
\begin{itemize}[parsep=0.5ex]
  \item 编程语言: Python > C/C++ > Java
  \item 平台: Linux
  \item 机器学习框架: Tensorflow
  \item 语言: 英语 - 熟练(四级600, 六级581, TOEFL 95)

\end{itemize}

\section{获奖情况}
\datedline{研究生国家奖学金}{2018.09}
\datedline{北京市优秀毕业生}{2017.07}

%% Reference
%\newpage
%\bibliographystyle{IEEETran}
%\bibliography{mycite}
\end{document}
