% !TEX TS-program = xelatex
% !TEX encoding = UTF-8 Unicode
% !Mode:: "TeX:UTF-8"

\documentclass{resume}
\usepackage{zh_CN-Adobefonts_external} % Simplified Chinese Support using external fonts (./fonts/zh_CN-Adobe/)
% \usepackage{NotoSansSC_external}
% \usepackage{NotoSerifCJKsc_external}
% \usepackage{zh_CN-Adobefonts_internal} % Simplified Chinese Support using system fonts
\usepackage{linespacing_fix} % disable extra space before next section
\usepackage{cite}

\newlist{InlineList}{enumerate*}{1}
\setlist[InlineList]{label=(\roman*)}


\begin{document}
\pagenumbering{gobble} % suppress displaying page number

\name{梅家洁}

\basicInfo{
  \email{jiajie.mei@buaa.edu.cn} \textperiodcentered\ 
  \phone{(+86) 188 1004 2048} %\textperiodcentered\ 
  %\linkedin[billryan8]{https://www.linkedin.com/in/billryan8}
  }
 
\section{教育背景}
\datedsubsection{\textbf{北京航空航天大学}}{2017.09 -- 2020.01}
\textit{硕士}\ 生成模型与知识图谱
\datedsubsection{\textbf{北京航空航天大学}}{2013.09 -- 2017.07}
\textit{学士}\ 计算机科学与技术,保研综合排名29/238


\section{科研经历}
\datedsubsection{\textbf{计算机新技术研究所 (ACT)}, 北航,导师:张日崇\& Yongyi Mao}{2016.08 -- 2018.12}
\begin{itemize}
	\item 知识图谱(已发表):现有的KB embedding模型在执行实体预测任务时采用的是top-$k$准则,我观察到该准则不能在所有的实体预测任务上都取得均衡的查准率 (P) 和查全率 (R)。我提出使用max-$k$准则,并且指出对于理想模型,max-$k$准则的P, R, F1都至少和top-$k$的一样好。我提出了一个符合max-$k$的预测协议,即采样协议。我对采样协议进行渐近分析,导出了另一个max-$k$协议——贪心协议。该工作发表在SIGIR2018 (On Link Prediction in Knowledge Bases: Max-K Criterion and Prediction Protocols) (长文)。
\item 生成模型(探索性):
\begin{InlineList}
\item 我深入理解Variational AutoEncoder (VAE),基于它做了一些探索性研究。例如,我初步研究过VAE用于句子生成时常会出现KL散度为零问题。相比现有的一些经验性方法,我尝试推导VAE优化目标的最优解。为此,我构造了一个最简单的线性VAE,推导得到KL为零所对应的解确实是一个驻点。
\item 我了解GAN,基于它也做过一些探索性研究。我曾经想基于GAN构造一个句子打分器。在初步研究中,我构造了一个二维的玩具数据集,使用多个判别器与一个生成器对抗。但是最终生成器总是能近似生成玩具数据,判别器会失效。后来我意识到,这是GAN的优化目标必定会导致的情况。
\end{InlineList}


\end{itemize}

\section{实习经历}
\datedsubsection{\textbf{NLP中心}, 美团点评,导师:王金刚\&张富峥}{2018.12 -- 2019.03}
\begin{itemize}
\item 推荐理由生成(已投稿):我使用Hive, Hadoop, Spark等工具从点评的搜索日志中导出数据集。我尝试了Pointer-generator模型(用户评论作为原文,推荐理由作为摘要),并且发现在encoder端引入selective gate能进一步提升ROUGE性能。用户查询则通过BiLSTM学习表示,并引入到encoder和decoder中。实验结果表明该模型更能生成查询相关的推荐理由。该工作已投稿IJCAI2019 (Query-Aware Tips Generation for Vertical Search)。

\end{itemize}


\section{技能}
% increase linespacing [parsep=0.5ex]
\begin{itemize}[parsep=0.5ex]
  \item 编程语言: Python > C/C++ > Java
  \item 平台: Linux
  \item 机器学习框架: Tensorflow
  \item 语言: 英语 - 熟练(四级600, 六级581, TOEFL 95)

\end{itemize}

\section{获奖情况}
\datedline{研究生国家奖学金}{2018.09}
\datedline{北京市优秀毕业生}{2017.07}
\datedline{第七届蓝桥杯大赛北京赛区C/C++程序设计大赛A组二等奖}{2016.03}

%% Reference
%\newpage
%\bibliographystyle{IEEETran}
%\bibliography{mycite}
\end{document}
